\subsection{Исходное изображение}

Для последнего задания выберем следующее изображение:

\begin{figure}[ht!]
    \centering
    \includegraphics[width=0.8\textwidth]{images/source_images/snooker.jpg}
    \caption{Шары для игры в бильярд}
    \label{img:billiard_balls}
\end{figure} 

\subsection{Программа на языке Python}
\input{code_segmentation.tex}

\subsection{Результат}

Для начала необходимо получить бинарное изображение, предварительно удалив лишние детали.

\begin{figure}[ht!]
    \centering
    \includegraphics[width=0.7\textwidth]{images/transformed_images/3/Binary_closed.jpg}
    \caption{Бинарное изображение}
    \label{img:bin_balls}
\end{figure} 

Далее воспользуемся преобразованием евклидова расстояния и после дополнительной фильтрации получим маркеры переднего плана:

\begin{figure}[ht!]
    \centering
    \begin{subfigure}{0.4\textwidth}
        \includegraphics[width=\textwidth]{images/transformed_images/3/Fg.jpg}
        \caption{}
        \label{img:bin_fg}
    \end{subfigure}
    \begin{subfigure}{0.4\textwidth}
        \includegraphics[width=\textwidth]{images/transformed_images/3/Jet_fg_markers.jpg}
        \caption{}
        \label{img:jet_fg}
    \end{subfigure}
    \caption{Определение маркеров переднего плана: (a) область маркеров переднего плана, (b) маркеры переднего плана}
    \label{img:FG}
\end{figure} 

Далее получим маркеры фона и неопределенной области:

\begin{figure}[ht!]
    \centering
    \begin{subfigure}{0.4\textwidth}
        \includegraphics[width=\textwidth]{images/transformed_images/3/Bg.jpg}
        \caption{}
        \label{img:bin_bg}
    \end{subfigure}
    \begin{subfigure}{0.4\textwidth}
        \includegraphics[width=\textwidth]{images/transformed_images/3/UND.jpg}
        \caption{}
        \label{img:bin_und}
    \end{subfigure}
    \caption{Определение маркеров: (a) фона, (b) неопределенной области}
    \label{img:BG}
\end{figure} 

Наконец, получим результаты сегментации:

\begin{figure}[ht!]
    \centering
    \includegraphics[width=0.7\textwidth]{images/transformed_images/3/Jet.jpg}
    \caption{Результат сегментации}
    \label{img:seg}
\end{figure} 
\clearpage
\begin{figure}[ht!]
    \centering
    \includegraphics[width=0.7\textwidth]{images/transformed_images/3/result.jpg}
    \caption{Результат сегментации, наложенный на исходное изображение}
    \label{img:seg_orig}
\end{figure} 